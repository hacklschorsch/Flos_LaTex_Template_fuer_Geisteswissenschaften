\documentclass[a4paper,12pt,pdftex,oneside,halfparskip]{scrreprt}

% Font von http://www.tug.dk/FontCatalogue/garamond/
% Zur Installation siehe auch (Shell-Befehl: "getnonfreefonts garamond classico")
% http://my.opera.com/freedo/blog/2008/09/04/wie-kommt-die-garamond-ins-latex
% und http://gael-varoquaux.info/computers/garamond/index.html
%\usepackage[garamond]{}
%\linespread {1.08}   % Garamond needs more leading (space between lines)
% you should never, ever, use the standards out-of-the-box T1 fonts with
% pdfLaTeX, they look ugly.Always include the "lmodern" or "pslatex" package,
% that uses much better postscript fonts.
%\usepackage[lmodern]{}
%\usepackage[pslatex]{}

\usepackage[T1]{fontenc}
\usepackage[utf8]{inputenc}
\usepackage[ngerman]{babel}

% Schrifttypen umstellen,
% http://my.opera.com/freedo/blog/2008/09/04/wie-kommt-die-garamond-ins-latex
\renewcommand{\rmdefault}{ugm}
\renewcommand{\sfdefault}{uop}
\renewcommand{\ttdefault}{pcr}

% Schriften benutzen geht mit sowas wie "\rmfamily":
% {\rmfamily URW Garamond}
% {\sffamily URW Classico a.k.a. Optima}
% {\ttfamily URW Nimbus Mono a.k.a. Courier}


\usepackage{csquotes}

% apabackref=true ist kein APA style, aber praktisch 
\usepackage[backend=biber, style=apa, apabackref=true, uniquename=false]{biblatex}
\DeclareLanguageMapping{ngerman}{ngerman-apa}
\addbibresource{thesis.bib}

\usepackage{url}
\usepackage[pdftex,
		unicode=true,
		colorlinks=false,
		pdfborder={0 0 0},
		pdfpagelabels=true,
		hyperindex=true]{hyperref}
\usepackage[automark]{scrlayer-scrpage}
\KOMAoptions{headsepline}
\usepackage{graphicx}
\usepackage{tabularx}
\usepackage{booktabs}
\usepackage{geometry}
\usepackage{makeidx}
\usepackage{rotating}

\usepackage{paralist}

\usepackage[printonlyused]{acronym}
\usepackage{xspace}

% Set font to 'Optima' for title page(s)
%\renewcommand{\rmdefault}{uop}

%% Define a new 'leo' style for the package that will use a smaller font.
\makeatletter
\def\url@leostyle{%
  \@ifundefined{selectfont}{\def\UrlFont{\sf}}{\def\UrlFont{\small}}}
\makeatother
%% Now actually use the newly defined style.
\urlstyle{leo}

\renewcommand{\arraystretch}{1.15}


% Eigene Befehle definieren:
\newcommand{\zB}{\mbox{z.\,B.}\xspace}
\newcommand{\iTitle}{Über den Nutzen eines ausgereiften Textsatzsystems für Arbeiten mit mehr als fünf Seiten}
\newcommand{\iAuthor}{Daniel Düsentrieb}


\title{\iTitle{}}
\author{\iAuthor{}}
\date{\today}
%\dedication{Für Mutti}

%\definecolor{darkblue}{rgb}{0,0,.1}
\hypersetup{
%	colorlinks=false, 
%	breaklinks=true, 
%	linkcolor=darkblue,
%	menucolor=darkblue,
%	urlcolor=darkblue,
%	citecolor=darkblue,
    pdftitle={\iTitle{}},
    pdfauthor={\iAuthor{}},
    pdfsubject={Masterarbeit},  % subject of the document
    pdfcreator={\iAuthor{}},  % creator of the document
    pdfproducer={\iAuthor{}}, % producer of the document
    pdfkeywords={Ein} {zwei} {Stichwörter}
}

\graphicspath{{images/}}

% Keine "Schusterjungen"
\clubpenalty = 10000
% Keine "Hurenkinder"
\widowpenalty = 10000 \displaywidowpenalty = 10000
\setlength{\baselineskip}{3ex}     

\makeindex


% Trennvorschläge:
\hyphenation{Donau-dampf-schiff}
\hyphenation{con-ven-tion-al-ized}
\hyphenation{Ge-sprächs-part-ners}
\hyphenation{ei-nen}
\hyphenation{nicht-re-dun-dan-te}
\hyphenation{re-dun-dan-te}
\hyphenation{pho-no-lo-gi-sche}




\begin{document}


  \begin{center}
  \thispagestyle{empty}

  \bigskip \bigskip \bigskip 
  \vspace{6.0cm} \ \\ % Uni Logo hier rein
  \includegraphics[height=6cm]{images/Wikiversity-logo-es.pdf} \\
  \vspace*{2.0cm}
  {\huge \bf Wikiversität} \\
  {\Large \bf Zentrum für Wikis und Versitäten} \\
  \bigskip \bigskip \bigskip
  {\Large \bf Fachgebiet des Lehrstuhls} \\
  Dipl. M.A. angestreber Abschluss
  \bigskip \bigskip \bigskip

  {\Large Abschlussarbeit zur Erlangung des akademischen Grades (Grad hier einfügen)}\\
   \bigskip\bigskip\bigskip
  {\Large \iAuthor{}} \\
  \bigskip\bigskip
  \noindent\textbf{\Large 
  {\iTitle{}}}

  \end{center}

\cleardoublepage

  \begin{center}

  \bigskip \bigskip \bigskip 
  \vspace{6.0cm} \ \\ % Uni Logo hier rein
  \vspace*{0.8cm}
  {\huge \bf Wikiversität} \\
  {\Large \bf Zentrum für Wikis und Versitäten} \\
  \bigskip \bigskip \bigskip
  {\Large \bf Fachgebiet des Lehrstuhls} \\
  Dipl. M.A. angestreber Abschluss
  \bigskip \bigskip \bigskip

  {\Large Abschlussarbeit zur Erlangung des akademischen Grades (Grad hier einfügen)}\\
   \bigskip\bigskip\bigskip
  {\Large \iAuthor{}} \\

Straßenstr. 123 \\
80337 München \\
e.mail@adresse.de\\
Sechstes Fachsemester

  \bigskip \bigskip

  \noindent\textbf{\Large 
  \iTitle{}}

\bigskip\bigskip\bigskip\bigskip\bigskip
\bigskip

  \vfill

\begin{tabularx}{\textwidth}{Xr}
 Erstgutachter: & Dr.\,Meister Eder\\
 Zweitgutachter: & Dr.\,Michael Jackson\\
 Datum: & 30. Juni 2012 \\

\end{tabularx}
\end{center}

\newpage
\ \\

\renewcommand{\rmdefault}{ugm}
\rmfamily

\vfill

Ich versichere, dass ich diese Masterarbeit selbständig verfasst und nur die angegebenen Quellen und Hilfsmittel verwendet habe. 

\vspace{1.5cm}

\begin{flushleft}\footnotesize
\begin{tabular}{@{}p{0.3\linewidth}p{0.2\linewidth}>{\centering\arraybackslash}p{0.4\linewidth}}
& & \\ \cline{3-3}
& & Datum, Unterschrift \\ [4\baselineskip]
\end{tabular}
\end{flushleft}

\newpage
\ \\
\vfill

% Seitenzahlen (Ziffern), Seite 1 ab hier.
\setcounter{page}{1}
\pagenumbering{roman}


Ich danke meiner Mama, meinem Papa und der lieben Frau Doktor.
Ich danke meiner Mama, meinem Papa und der lieben Frau Doktor.
Ich danke meiner Mama, meinem Papa und der lieben Frau Doktor.
Ich danke meiner Mama, meinem Papa und der lieben Frau Doktor.




\vspace{9.5cm}

\cleardoublepage

% input statt include, sonst kriegen wir eine seitenzahl.
\chapter*{Abstract}

Zusammenfassung auf Deutsch: In dieser Arbeit wird die seit langem viel diskutierte argumentative Lücke endlich geschlossen.
Die Forschungshöhe ist sofort ersichtlich.
Zusammenfassung auf Deutsch: In dieser Arbeit wird die seit langem viel diskutierte argumentative Lücke endlich geschlossen.
Die Forschungshöhe ist sofort ersichtlich.
Zusammenfassung auf Deutsch: In dieser Arbeit wird die seit langem viel diskutierte argumentative Lücke endlich geschlossen.
Die Forschungshöhe ist sofort ersichtlich.

\bigskip\bigskip

Abstract en ingles.
Abstract in English.
Work really hard on this section as it's the only thing ppl will read.
Well, maybe they'll also read the conclusion, but only if your abstract strikes them as excellent.


\input{vorwort}


\pagebreak

%\setcounter{tocdepth}{3}  % wie tief soll das inhaltsverzeichnis so sein
%\addtocontents{toc}
\tableofcontents

\cleardoublepage


\KOMAoption{listof}{leveldown}
\addchap*{Abbildungs- und Tabellenverzeichnis}
\listoffigures
\listoftables % Rausnehmen, falls es keine Tabelle gibt.
\cleardoublepage


% Seitenzahlen (Ziffern), Seite 1 ab hier.
%\setcounter{page}{1}
\pagenumbering{arabic}

\chapter{Einleitung}

Dies ist die Einleitung, die zum Inhalt der Arbeit hinführt.
Sie sollte wahrscheinlich für jemand Fachfremden mit universitärer Vorbildung verständlich sein.

Ziel der vorliegenden Arbeit ist es, Ergebnisse der Forschung in die Praxis einzuführen und auf diese Weise fundiert die Arbeit zu begründen und zu beschreiben, sowie die Bedeutung für das Lernen hervorzuheben.
Ziel der vorliegenden Arbeit ist es, Ergebnisse der Forschung in die Praxis einzuführen und auf diese Weise fundiert die Arbeit zu begründen und zu beschreiben, sowie die Bedeutung für das Lernen hervorzuheben.
Ziel der vorliegenden Arbeit ist es, Ergebnisse der Forschung in die Praxis einzuführen und auf diese Weise fundiert die Arbeit zu begründen und zu beschreiben, sowie die Bedeutung für das Lernen hervorzuheben.
Ziel der vorliegenden Arbeit ist es, Ergebnisse der Forschung in die Praxis einzuführen und auf diese Weise fundiert die Arbeit zu begründen und zu beschreiben, sowie die Bedeutung für das Lernen hervorzuheben.


\chapter{Zweites Kapitel}

Das hier ist ein Beispiel, wie die Hierarchie Chapter / Section / Subsection / Subsubsection / Paragraph dargestellt wird.

\section{Section}

Das hier ist etwas Fliesztext einer Section.
Das hier ist etwas Fliesztext einer Section.
Das hier ist etwas Fliesztext einer Section.
Das hier ist etwas Fliesztext einer Section.

\subsection{Subsection}

Das hier ist etwas Fliesztext einer Subsection.
Das hier ist etwas Fliesztext einer Subsection.
Das hier ist etwas Fliesztext einer Subsection.
Das hier ist etwas Fliesztext einer Subsection.
Das hier ist etwas Fliesztext einer Subsection.

\subsubsection{Subsubsection}

Das hier ist etwas Fliesztext einer Subsubsection.
Das hier ist etwas Fliesztext einer Subsubsection.
Das hier ist etwas Fliesztext einer Subsubsection.
Das hier ist etwas Fliesztext einer Subsubsection.
Das hier ist etwas Fliesztext einer Subsubsection.
Das hier ist etwas Fliesztext einer Subsubsection.

\paragraph{Paragraph}

Das hier ist etwas Fliesztext eines Paragraphs.
Das hier ist etwas Fliesztext eines Paragraphs.
Das hier ist etwas Fliesztext eines Paragraphs.
Das hier ist etwas Fliesztext eines Paragraphs.
Das hier ist etwas Fliesztext eines Paragraphs.
Das hier ist etwas Fliesztext eines Paragraphs.

\begin{itemize}
\item Erster Punkt
\item Zweiter Punkt
\item Dritter Punkt
\end{itemize}

\begin{enumerate}
\item Punkt
\item Punkt
\item Punkt
\end{enumerate}

\section{So wird BibLatex und Apacite benutzt}

Hier kann nur ein sehr knappes Beispiel.
Bitte die exzellente Dokumentation von ApaCite wenigstens überfliegen.

\glqq{}Gesten sind kommunikative Bewegungen der Hände und Arme\grqq{}, so \Textcite[13]{Muller.1998}.

Oder auch als Block. Das kommt ganz gut bei längeren Zitaten. Beispiel:

\begin{quote}
\glqq{}Ich kann tote Menschen sehen. Ich kann tote Menschen sehen. Ich kann tote Menschen sehen. Ich kann tote Menschen sehen. Ich kann tote Menschen sehen. Ich kann tote Menschen sehen. Ich kann tote Menschen sehen.\grqq{} \Parencites[1]{Muller.1998}[17]{Muller.1998}
\end{quote}


So bindet man ein Bild ein.
Figur \ref{img_wikiversidad} wird nicht unbedingt genau dort eingebunden wo der Text steht, sondern irgendwo in der Nähe.
Das ist natürlich alles konfigurierbar.

\begin{figure}
\begin{center}
  \includegraphics[height=8cm]{Wikiversity-logo-es}
\end{center}
\caption{Logo der Wikiversität Spanien \parencite[23]{Muller.1998}}
\label{img_wikiversidad}
\end{figure}

% und % beginnt einen kommentar, diese Zeile landet dann nicht in der PDF.
% kann man gut nehmen um notizen zu machen im dokument:
% TODO hier muss noch etwas sinnvolles her

Tabellen sind etwas gewöhungsbedürftig finde ich, siehe Tabelle \ref{tab_beispiel1}.

\begin{table}[h]
\begin{center}
\begin{tabular}{lll}
Erste Spalte & Zweite Spalte & dritte Spalte \\
Erste Spalte & Zweite Spalte & dritte Spalte \\
Erste Spalte & Zweite Spalte & dritte Spalte \\
\end{tabular}
\end{center}
\caption{Beispieltabelle \parencite[nach][2]{Muller.1998}}
\label{tab_beispiel1}
\end{table}


\chapter{Resümee und Ausblick}

Ein nicht minder wichtiger Teil ist der Schluss, in dem die Ergebnisse der vorliegenden Arbeit kurz zusammengefasst werden und so dem interessiertem Leser -- der bisher mit annähernder Sicherheit nur den Abstract gelesen hat -- zu erklären, dass die Zeit des Autors wie auch der Leserschaft gut investiert ist.
Ein nicht minder wichtiger Teil ist der Schluss, in dem die Ergebnisse der vorliegenden Arbeit kurz zusammengefasst werden und so dem interessiertem Leser -- der bisher mit annähernder Sicherheit nur den Abstract gelesen hat -- zu erklären, dass die Zeit des Autors wie auch der Leserschaft gut investiert ist.



\cleardoublepage

\appendix

\clearpage

\raggedright
\printbibliography 

\end{document}
